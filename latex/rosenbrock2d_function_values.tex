
The minimal. maximal, mean and median values of the solutions are shown in Table~\ref{function_values:rosenbrock2d}:

\begin{table}[H]
\centering
\caption{Statistical Information about function values For Rosenbrock Function}
\label{function_values:rosenbrock2d}
\begin{tabular}{lrrrr}
\toprule
 Alg. &  Min &  Max &  Mean &  Median \\
\midrule
  dfp & 0.00 & 1.33 &  0.04 &    0.00 \\
 bfgs & 0.00 & 0.00 &  0.00 &    0.00 \\
lbfgs & 0.00 & 0.00 &  0.00 &    0.00 \\
  lma & 0.00 & 0.00 &  0.00 &    0.00 \\
\bottomrule
\end{tabular}
\end{table}

The values again are rounded. The median informs us that, for all
algorithms is expected to find the minimun of Rosenbrock Function.
Although the mean of DFP algorithm is greater than zero, it is expected
to converge more than diverge in the function constrained search space.