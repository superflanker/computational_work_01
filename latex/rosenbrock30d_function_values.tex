
The minimal. maximal, mean and median values of the solutions are shown in Table~\ref{function_values:rosenbrock30d}:

\begin{table}[H]
\centering
\caption{Statistical Information about function values For Rosenbrock Function}
\label{function_values:rosenbrock30d}
\begin{tabular}{lrrrr}
\toprule
 Alg. &  Min &   Max &  Mean &  Median \\
\midrule
  dfp & 0.00 & 29.25 & 13.10 &   12.58 \\
 bfgs & 0.00 &  3.99 &  0.80 &    0.00 \\
lbfgs & 0.00 &  3.99 &  0.20 &    0.00 \\
  lma & 0.00 &  3.99 &  0.48 &    0.00 \\
\bottomrule
\end{tabular}
\end{table}

The median value informs us that all algorithms are like to find the solution
for Rosenbrock Function in 4 dimesions. The cases of poor convergence that all algorithms
have are reflected in the mean value. Looking at the Maximum value, it is noted that the mean value is
quite the same percentage shown in Table~\ref{convergence:rosenbrock30d}.
