
The convergence report for the Rosenbrock Function in 30 dimensions is shown in Table~\ref{convergence:rosenbrock30d}:

\begin{table}[H]
\centering
\caption{Convergence Report For Rosenbrock Function}
\label{convergence:rosenbrock30d}
\begin{tabular}{lrrrr}
\toprule
 Alg. &  Good &  Poor &  Diver. &  Total \\
\midrule
  dfp &     1 &    99 &       0 &    100 \\
 bfgs &    80 &    20 &       0 &    100 \\
lbfgs &    95 &     5 &       0 &    100 \\
  lma &    88 &    12 &       0 &    100 \\
\bottomrule
\end{tabular}
\end{table}

DFP has poor convergence properties for the Rosenbrock Function with more than 2 dimensions.
The rest have quite the same profile regarding convergence. It is important to note that
although all algorithms does not converge for the presented initial solutions, it does not
diverge at all. Because of the stop criterion of all algorithms, which is the function gradient value,
and the banana shaped valley (plateau) that Rosenbrock function has, we have this behaviour.