\documentclass{ieeeaccess}
\usepackage{cite}
\usepackage{amsmath,amssymb,amsfonts}
\usepackage{algorithmic}
\usepackage{graphicx}
\usepackage{textcomp}
\usepackage{booktabs}
\def\BibTeX{{\rm B\kern-.05em{\sc i\kern-.025em b}\kern-.08em
    T\kern-.1667em\lower.7ex\hbox{E}\kern-.125emX}}
\begin{document}
\history{Unpublished, First Preview}
\doi{Unpublished/NoDOIInformation}
\title{Quasi-Newton Methods Comparision using Ackley, Beale, Booth, Matyas, Rastrigin and RosenBrock Functions}
\author{
	\IEEEauthorblockN{Augusto Mathias Adams\IEEEauthorrefmark{1}}
    \IEEEauthorblockN{Caio Phillipe Mizerkowski\IEEEauthorrefmark{2}}
    \IEEEauthorblockN{Christian Piltz Araújo\IEEEauthorrefmark{3}}
    \IEEEauthorblockN{Vinicius Eduardo dos Reis\IEEEauthorrefmark{4}}
	\IEEEauthorblockA{\IEEEauthorrefmark{1}GRR20172143, augusto.adams@ufpr.br}
	\IEEEauthorblockA{\IEEEauthorrefmark{2}GRR20166403, caiomizerkowski@gmail.com}
	\IEEEauthorblockA{\IEEEauthorrefmark{3}GRR20172197, christian0294@yahoo.com.br}
	\IEEEauthorblockA{\IEEEauthorrefmark{4}GRR20175957, eduardo.reis02@gmail.com}
}

\markboth
{Adams \headeretal: Quasi-Newton Methods Comparision using Ackley, Beale, Booth, Matyas, Rastrigin and RosenBrock Functions}
{Adams \headeretal: Quasi-Newton Methods Comparision using Ackley, Beale, Booth, Matyas, Rastrigin and RosenBrock Functions}

\corresp{Corresponding author: Augusto Mathias Adams (e-mail: augusto.adams@ufpr.br).}
\begin{abstract}
This paper discusses briefly four popular algorithms to solve optimization/minimization problems, beloginging to Quasi-Newton class: \textit{Davidon–Fletcher–Powell} (DFP), \textit{Broyden–Fletcher–Goldfarb–Shanno} (BFGS), limited memory BFGS and \textit{Levenberg-Marquardt Algorithm} (LMA). These algorithms are implemented in \textit{Python language}, version 3.10 and uses \textit{SymPy}, \textit{SciPy} and \textit{NumPy} libraries. One of them, \textit{BFGS}, is implemented natively on the  \textit{SciPy} library and the rest are implemented by hand to provide useful insights about the inner operation of the algorithms. The results are, however, very surprising with a few exceptions: when the initial solution is placed in a region near the global minimum, almost all algorithms converges very quickly, so, a study was made to limit the initial solution to  these regions. With exception of 
\end{abstract}

\begin{keywords} 
Optimization Methods, Quasi-Newton, Non-linear Programming
\end{keywords}

\titlepgskip=-15pt

\maketitle

\section{Introduction}
\label{sec:introduction}