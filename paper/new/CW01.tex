\documentclass[conference]{IEEEtran}
\IEEEoverridecommandlockouts
% The preceding line is only needed to identify funding in the first footnote. If that is unneeded, please comment it out.
\usepackage{cite}
\usepackage{amsmath,amssymb,amsfonts}
\usepackage{mathtools}
\usepackage{algpseudocode}
\usepackage{algorithm}
\usepackage{algorithmicx}
\usepackage{microtype}
\usepackage{float}
\usepackage{adjustbox}
\usepackage{booktabs,makecell,tabularx}
\usepackage{graphicx}
\usepackage{textcomp}
\usepackage{verbatim}
\usepackage{xcolor}
\usepackage{graphics} % for pdf, bitmapped graphics files
\usepackage{epsfig} % for postscript graphics files
\usepackage{mathptmx} % assumes new font selection scheme installed
\usepackage{multicol}
\usepackage[english]{babel}
\usepackage[T1]{fontenc}

\def\BibTeX{{\rm B\kern-.05em{\sc i\kern-.025em b}\kern-.08em
    T\kern-.1667em\lower.7ex\hbox{E}\kern-.125emX}}
\newcommand\norm[1]{\left\Vert#1\right\Vert}
\begin{document}

\title{Quasi-Newton Methods Comparision using Ackley, Beale, Booth, Matyas, Rastrigin and RosenBrock Functions}
\author{
	\IEEEauthorblockN{Augusto Mathias Adams\IEEEauthorrefmark{1}, Caio Phillipe Mizerkowski\IEEEauthorrefmark{2}, Christian Piltz Araújo\IEEEauthorrefmark{3}, Vinicius Eduardo dos Reis\IEEEauthorrefmark{4}}
	\IEEEauthorblockA{\IEEEauthorrefmark{1}GRR20172143 - augusto.adams@ufpr.br, \IEEEauthorrefmark{2}GRR20166403 -  caiomizerkowski@gmail.com} \IEEEauthorblockA{\IEEEauthorrefmark{3}GRR20172197 - christian0294@yahoo.com.br, \IEEEauthorrefmark{4}GRR20175957 - eduardo.reis02@gmail.com}
}

\markboth
{Adams \headeretal: Quasi-Newton Methods Comparision using Ackley, Beale, Booth, Matyas, Rastrigin and RosenBrock Functions}
{Adams \headeretal: Quasi-Newton Methods Comparision using Ackley, Beale, Booth, Matyas, Rastrigin and RosenBrock Functions}


\maketitle

\begin{abstract}
This paper discusses briefly four popular algorithms to solve optimization/minimization problems, beloginging to Quasi-Newton class: \textit{Davidon–Fletcher–Powell} (DFP), \textit{Broyden–Fletcher–Goldfarb–Shanno} (BFGS), limited memory BFGS and \textit{Levenberg-Marquardt Algorithm} (LMA). These algorithms are implemented in \textit{Python language}, version 3.10 and uses \textit{SymPy}, \textit{SciPy} and \textit{NumPy} libraries. One of them, \textit{BFGS}, is implemented natively on the  \textit{SciPy} library and the rest are implemented by hand to provide useful insights about the inner operation of the algorithms. The results are, however, very corrent with a few exceptions.
\end{abstract}


\begin{IEEEkeywords}Optimization Problem; Linear Programming; Non-linear Programming; Quasi-Newton Algorithms\end{IEEEkeywords}

\section{Introduction}

\section{Methodology}

\section{Results}

\subsection{Ackley Function}

\begin{table}
\centering
\caption{Best Fits For Ackley Function}
\label{solutions:ackley}
\begin{tabular}{llrrr}
\toprule
 Alg. &    Sol. &  Iter. &  F. Eval &  F. Value \\
\midrule
  dfp & $S_{1}$ &     36 &      108 &      0.00 \\
 bfgs & $S_{2}$ &     42 &      129 &      0.00 \\
lbfgs & $S_{3}$ &     37 &      389 &      0.00 \\
  lma & $S_{4}$ &     38 &      320 &      0.00 \\
\bottomrule
\end{tabular}
\end{table}


The minimal. maximal, mean and median values of the solutions are shown in Table~\ref{function_values:ackley}:

\begin{table}[H]
\centering
\caption{Statistical Information about function values For Ackley Function}
\label{function_values:ackley}
\begin{tabular}{lrrrr}
\toprule
 Alg. &  Min &  Max &  Mean &  Median \\
\midrule
  dfp & 0.00 & 0.00 &  0.00 &    0.00 \\
 bfgs & 0.00 & 0.00 &  0.00 &    0.00 \\
lbfgs & 0.00 & 0.00 &  0.00 &    0.00 \\
  lma & 0.00 & 0.00 &  0.00 &    0.00 \\
\bottomrule
\end{tabular}
\end{table}

The values again are rounded. The matter is the mdian value: it informs us the result that
we are like to obtain when running any algorithm to solve for minimum. As it seems, for all
algorithms is expected to find the minimun of Ackley Function.



\end{document}
